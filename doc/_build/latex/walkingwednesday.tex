%% Generated by Sphinx.
\def\sphinxdocclass{report}
\documentclass[letterpaper,10pt,english]{sphinxmanual}
\ifdefined\pdfpxdimen
   \let\sphinxpxdimen\pdfpxdimen\else\newdimen\sphinxpxdimen
\fi \sphinxpxdimen=.75bp\relax
\ifdefined\pdfimageresolution
    \pdfimageresolution= \numexpr \dimexpr1in\relax/\sphinxpxdimen\relax
\fi
%% let collapsible pdf bookmarks panel have high depth per default
\PassOptionsToPackage{bookmarksdepth=5}{hyperref}

\PassOptionsToPackage{booktabs}{sphinx}
\PassOptionsToPackage{colorrows}{sphinx}

\PassOptionsToPackage{warn}{textcomp}
\usepackage[utf8]{inputenc}
\ifdefined\DeclareUnicodeCharacter
% support both utf8 and utf8x syntaxes
  \ifdefined\DeclareUnicodeCharacterAsOptional
    \def\sphinxDUC#1{\DeclareUnicodeCharacter{"#1}}
  \else
    \let\sphinxDUC\DeclareUnicodeCharacter
  \fi
  \sphinxDUC{00A0}{\nobreakspace}
  \sphinxDUC{2500}{\sphinxunichar{2500}}
  \sphinxDUC{2502}{\sphinxunichar{2502}}
  \sphinxDUC{2514}{\sphinxunichar{2514}}
  \sphinxDUC{251C}{\sphinxunichar{251C}}
  \sphinxDUC{2572}{\textbackslash}
\fi
\usepackage{cmap}
\usepackage[T1]{fontenc}
\usepackage{amsmath,amssymb,amstext}
\usepackage{babel}



\usepackage{tgtermes}
\usepackage{tgheros}
\renewcommand{\ttdefault}{txtt}



\usepackage[Bjarne]{fncychap}
\usepackage{sphinx}

\fvset{fontsize=auto}
\usepackage{geometry}


% Include hyperref last.
\usepackage{hyperref}
% Fix anchor placement for figures with captions.
\usepackage{hypcap}% it must be loaded after hyperref.
% Set up styles of URL: it should be placed after hyperref.
\urlstyle{same}

\addto\captionsenglish{\renewcommand{\contentsname}{Contents:}}

\usepackage{sphinxmessages}
\setcounter{tocdepth}{1}



\title{Walking Wednesday}
\date{Mar 06, 2024}
\release{1.0.0}
\author{Andy Wang}
\newcommand{\sphinxlogo}{\vbox{}}
\renewcommand{\releasename}{Release}
\makeindex
\begin{document}

\ifdefined\shorthandoff
  \ifnum\catcode`\=\string=\active\shorthandoff{=}\fi
  \ifnum\catcode`\"=\active\shorthandoff{"}\fi
\fi

\pagestyle{empty}
\sphinxmaketitle
\pagestyle{plain}
\sphinxtableofcontents
\pagestyle{normal}
\phantomsection\label{\detokenize{index::doc}}


\sphinxstepscope
\index{module@\spxentry{module}!countDown@\spxentry{countDown}}\index{countDown@\spxentry{countDown}!module@\spxentry{module}}\index{changeCountdownNumber() (in module countDown)@\spxentry{changeCountdownNumber()}\spxextra{in module countDown}}\phantomsection\label{\detokenize{countDown:module-countDown}}

\begin{fulllineitems}
\phantomsection\label{\detokenize{countDown:countDown.changeCountdownNumber}}
\pysigstartsignatures
\pysiglinewithargsret{\sphinxcode{\sphinxupquote{countDown.}}\sphinxbfcode{\sphinxupquote{changeCountdownNumber}}}{\sphinxparam{\DUrole{n}{classNum}}}{}
\pysigstopsignatures
\sphinxAtStartPar
This function checks for how many turns can be done after current pair
\begin{description}
\sphinxlineitem{Args:}
\sphinxAtStartPar
classNum (str): eg. “1”

\sphinxlineitem{Returns:}
\sphinxAtStartPar
int: how many turns remain to be executed theoretically

\end{description}

\end{fulllineitems}

\index{checkAllPaired() (in module countDown)@\spxentry{checkAllPaired()}\spxextra{in module countDown}}

\begin{fulllineitems}
\phantomsection\label{\detokenize{countDown:countDown.checkAllPaired}}
\pysigstartsignatures
\pysiglinewithargsret{\sphinxcode{\sphinxupquote{countDown.}}\sphinxbfcode{\sphinxupquote{checkAllPaired}}}{\sphinxparam{\DUrole{n}{classNum}}}{}
\pysigstopsignatures
\sphinxAtStartPar
This function checks if certain person has been paired for everyone
\begin{description}
\sphinxlineitem{Args:}
\sphinxAtStartPar
classNum (str): eg. “1”

\sphinxlineitem{Returns:}
\sphinxAtStartPar
boolean: True/False

\end{description}

\end{fulllineitems}

\index{readCountdownNumber() (in module countDown)@\spxentry{readCountdownNumber()}\spxextra{in module countDown}}

\begin{fulllineitems}
\phantomsection\label{\detokenize{countDown:countDown.readCountdownNumber}}
\pysigstartsignatures
\pysiglinewithargsret{\sphinxcode{\sphinxupquote{countDown.}}\sphinxbfcode{\sphinxupquote{readCountdownNumber}}}{\sphinxparam{\DUrole{n}{classNum}}}{}
\pysigstopsignatures
\sphinxAtStartPar
This function reads how many turns of pairing are avaliable until automatic refresh
\begin{description}
\sphinxlineitem{Args:}
\sphinxAtStartPar
classNum (str): eg. “1”

\sphinxlineitem{Returns:}
\sphinxAtStartPar
int: how many turns remain to be executed theoretically

\end{description}

\end{fulllineitems}


\sphinxstepscope
\index{module@\spxentry{module}!ExcelFunctions@\spxentry{ExcelFunctions}}\index{ExcelFunctions@\spxentry{ExcelFunctions}!module@\spxentry{module}}\index{ExcelToMiddle() (in module ExcelFunctions)@\spxentry{ExcelToMiddle()}\spxextra{in module ExcelFunctions}}\phantomsection\label{\detokenize{ExcelFunctions:module-ExcelFunctions}}

\begin{fulllineitems}
\phantomsection\label{\detokenize{ExcelFunctions:ExcelFunctions.ExcelToMiddle}}
\pysigstartsignatures
\pysiglinewithargsret{\sphinxcode{\sphinxupquote{ExcelFunctions.}}\sphinxbfcode{\sphinxupquote{ExcelToMiddle}}}{\sphinxparam{\DUrole{n}{classNum}}\sphinxparamcomma \sphinxparam{\DUrole{n}{placement}}}{}
\pysigstopsignatures
\sphinxAtStartPar
This function overwrites the middle.txt file with the updated array
\begin{description}
\sphinxlineitem{Args:}
\sphinxAtStartPar
classNum (str): eg. “1”
placement (str): eg. “database”, “middle”

\sphinxlineitem{Returns:}
\sphinxAtStartPar
middle\textless{}classNum\textgreater{}.txt have new data from excel

\end{description}

\end{fulllineitems}

\index{MiddletoExcel() (in module ExcelFunctions)@\spxentry{MiddletoExcel()}\spxextra{in module ExcelFunctions}}

\begin{fulllineitems}
\phantomsection\label{\detokenize{ExcelFunctions:ExcelFunctions.MiddletoExcel}}
\pysigstartsignatures
\pysiglinewithargsret{\sphinxcode{\sphinxupquote{ExcelFunctions.}}\sphinxbfcode{\sphinxupquote{MiddletoExcel}}}{\sphinxparam{\DUrole{n}{classNum}}\sphinxparamcomma \sphinxparam{\DUrole{n}{placement}}}{}
\pysigstopsignatures
\sphinxAtStartPar
This function overwrites middle\textless{}classNum\textgreater{}.txt file and student names to excel to create a table
\begin{description}
\sphinxlineitem{Args:}
\sphinxAtStartPar
classNum (str): eg. “1”
placement (str): eg. “database”, “middle”

\sphinxlineitem{Returns:}
\sphinxAtStartPar
excel spreadsheet is filled with studnet names and true/false data
display name to row 1 and colA (reade classes2\sphinxhyphen{}\textless{}\#\textgreater{}.txt)

\end{description}

\end{fulllineitems}

\index{openExcelFile() (in module ExcelFunctions)@\spxentry{openExcelFile()}\spxextra{in module ExcelFunctions}}

\begin{fulllineitems}
\phantomsection\label{\detokenize{ExcelFunctions:ExcelFunctions.openExcelFile}}
\pysigstartsignatures
\pysiglinewithargsret{\sphinxcode{\sphinxupquote{ExcelFunctions.}}\sphinxbfcode{\sphinxupquote{openExcelFile}}}{\sphinxparam{\DUrole{n}{classNum}}}{}
\pysigstopsignatures
\sphinxAtStartPar
This function opens the excel file from python
\begin{description}
\sphinxlineitem{Args:}
\sphinxAtStartPar
classNum (str): eg. “1”

\sphinxlineitem{Returns:}
\sphinxAtStartPar
excel opened

\end{description}

\end{fulllineitems}

\index{readExcel() (in module ExcelFunctions)@\spxentry{readExcel()}\spxextra{in module ExcelFunctions}}

\begin{fulllineitems}
\phantomsection\label{\detokenize{ExcelFunctions:ExcelFunctions.readExcel}}
\pysigstartsignatures
\pysiglinewithargsret{\sphinxcode{\sphinxupquote{ExcelFunctions.}}\sphinxbfcode{\sphinxupquote{readExcel}}}{\sphinxparam{\DUrole{n}{classNum}}}{}
\pysigstopsignatures
\sphinxAtStartPar
This function reads the excel spreadsheet True/False section
\begin{description}
\sphinxlineitem{Args:}
\sphinxAtStartPar
classNum (str): eg. “1”

\sphinxlineitem{Returns:}
\sphinxAtStartPar
boolean{[}{]}{[}{]}: true/false

\end{description}

\end{fulllineitems}


\sphinxstepscope
\index{module@\spxentry{module}!GUI@\spxentry{GUI}}\index{GUI@\spxentry{GUI}!module@\spxentry{module}}\index{MiddleToDatabase() (in module GUI)@\spxentry{MiddleToDatabase()}\spxextra{in module GUI}}\phantomsection\label{\detokenize{GUI:module-GUI}}

\begin{fulllineitems}
\phantomsection\label{\detokenize{GUI:GUI.MiddleToDatabase}}
\pysigstartsignatures
\pysiglinewithargsret{\sphinxcode{\sphinxupquote{GUI.}}\sphinxbfcode{\sphinxupquote{MiddleToDatabase}}}{\sphinxparam{\DUrole{n}{classNum}}}{}
\pysigstopsignatures
\sphinxAtStartPar
This function records the new pairs stored in middle to database
\begin{description}
\sphinxlineitem{Args:}
\sphinxAtStartPar
classNum (str): eg. “1”

\sphinxlineitem{Returns:}
\sphinxAtStartPar
add new pairings as True values to database

\end{description}

\end{fulllineitems}

\index{getTable() (in module GUI)@\spxentry{getTable()}\spxextra{in module GUI}}

\begin{fulllineitems}
\phantomsection\label{\detokenize{GUI:GUI.getTable}}
\pysigstartsignatures
\pysiglinewithargsret{\sphinxcode{\sphinxupquote{GUI.}}\sphinxbfcode{\sphinxupquote{getTable}}}{\sphinxparam{\DUrole{n}{classNum}}\sphinxparamcomma \sphinxparam{\DUrole{n}{placement}}}{}
\pysigstopsignatures
\sphinxAtStartPar
This function reads 2D array in databases/\textless{}classNum\textgreater{}.txt
\begin{description}
\sphinxlineitem{Args:}
\sphinxAtStartPar
classNum (str) : eg. “1”
placement (str) : eg. “database”

\sphinxlineitem{Returns:}
\sphinxAtStartPar
boolean{[}{]}{[}{]} : 2d array

\end{description}

\end{fulllineitems}

\index{readStudentName() (in module GUI)@\spxentry{readStudentName()}\spxextra{in module GUI}}

\begin{fulllineitems}
\phantomsection\label{\detokenize{GUI:GUI.readStudentName}}
\pysigstartsignatures
\pysiglinewithargsret{\sphinxcode{\sphinxupquote{GUI.}}\sphinxbfcode{\sphinxupquote{readStudentName}}}{\sphinxparam{\DUrole{n}{classNum}}}{}
\pysigstopsignatures
\sphinxAtStartPar
This function reads array in class\textless{}classNum\textgreater{}.txt
\begin{description}
\sphinxlineitem{Args:}
\sphinxAtStartPar
classNum (str): eg. “1”

\sphinxlineitem{Returns:}
\sphinxAtStartPar
string {[}{]}: 1d array

\end{description}

\end{fulllineitems}

\index{refreshPairingLabel() (in module GUI)@\spxentry{refreshPairingLabel()}\spxextra{in module GUI}}

\begin{fulllineitems}
\phantomsection\label{\detokenize{GUI:GUI.refreshPairingLabel}}
\pysigstartsignatures
\pysiglinewithargsret{\sphinxcode{\sphinxupquote{GUI.}}\sphinxbfcode{\sphinxupquote{refreshPairingLabel}}}{\sphinxparam{\DUrole{n}{label}}\sphinxparamcomma \sphinxparam{\DUrole{n}{window}}\sphinxparamcomma \sphinxparam{\DUrole{n}{newText}}}{}
\pysigstopsignatures
\sphinxAtStartPar
This funciton refreshes the label
\begin{description}
\sphinxlineitem{Args:}
\sphinxAtStartPar
label (tk.Label): 
window (tk.Tk()):
newText (String):

\end{description}

\end{fulllineitems}

\index{resetDatabase() (in module GUI)@\spxentry{resetDatabase()}\spxextra{in module GUI}}

\begin{fulllineitems}
\phantomsection\label{\detokenize{GUI:GUI.resetDatabase}}
\pysigstartsignatures
\pysiglinewithargsret{\sphinxcode{\sphinxupquote{GUI.}}\sphinxbfcode{\sphinxupquote{resetDatabase}}}{\sphinxparam{\DUrole{n}{classNum}}}{}
\pysigstopsignatures
\sphinxAtStartPar
This functions initializes the database and returns the initial number of turns left until automatic reset
\begin{description}
\sphinxlineitem{Args:}
\sphinxAtStartPar
classNum (str): eg. “1”

\sphinxlineitem{Returns:}
\sphinxAtStartPar
int: how many turns can be perfectly executed

\end{description}

\end{fulllineitems}

\index{resetMiddle() (in module GUI)@\spxentry{resetMiddle()}\spxextra{in module GUI}}

\begin{fulllineitems}
\phantomsection\label{\detokenize{GUI:GUI.resetMiddle}}
\pysigstartsignatures
\pysiglinewithargsret{\sphinxcode{\sphinxupquote{GUI.}}\sphinxbfcode{\sphinxupquote{resetMiddle}}}{\sphinxparam{\DUrole{n}{classNum}}}{}
\pysigstopsignatures
\sphinxAtStartPar
This function resets the 2D array in database
\begin{description}
\sphinxlineitem{Args:}
\sphinxAtStartPar
classNum (str): eg. “1”

\sphinxlineitem{Returns:}
\sphinxAtStartPar
overwrite a new (empty, without anyone being paired) 2D array

\end{description}

\end{fulllineitems}

\index{saveStudentList() (in module GUI)@\spxentry{saveStudentList()}\spxextra{in module GUI}}

\begin{fulllineitems}
\phantomsection\label{\detokenize{GUI:GUI.saveStudentList}}
\pysigstartsignatures
\pysiglinewithargsret{\sphinxcode{\sphinxupquote{GUI.}}\sphinxbfcode{\sphinxupquote{saveStudentList}}}{\sphinxparam{\DUrole{n}{fileName}}\sphinxparamcomma \sphinxparam{\DUrole{n}{theWidget}}}{}
\pysigstopsignatures
\sphinxAtStartPar
This function saves the studentList from app to class\textless{}classNum\textgreater{}.txt
\begin{description}
\sphinxlineitem{Args:}
\sphinxAtStartPar
fileName (str): eg. class\textless{}classNum\textgreater{}.txt
theWidget (tk.Text):

\sphinxlineitem{Returns:}
\sphinxAtStartPar
saves text from theWidget to class\textless{}classNum\textgreater{}.txt

\end{description}

\end{fulllineitems}


\sphinxstepscope
\index{module@\spxentry{module}!pairing@\spxentry{pairing}}\index{pairing@\spxentry{pairing}!module@\spxentry{module}}\index{checkValid() (in module pairing)@\spxentry{checkValid()}\spxextra{in module pairing}}\phantomsection\label{\detokenize{pairing:module-pairing}}

\begin{fulllineitems}
\phantomsection\label{\detokenize{pairing:pairing.checkValid}}
\pysigstartsignatures
\pysiglinewithargsret{\sphinxcode{\sphinxupquote{pairing.}}\sphinxbfcode{\sphinxupquote{checkValid}}}{\sphinxparam{\DUrole{n}{classNum}}\sphinxparamcomma \sphinxparam{\DUrole{n}{combo}}}{}
\pysigstopsignatures
\sphinxAtStartPar
This function checks if current combo has any repetition from previous pairs
\begin{description}
\sphinxlineitem{Args:}
\sphinxAtStartPar
classNum (str): eg. “1”
combo (2D array): eg. {[}{[}4,1{]},{[}7,10{]}….{]}

\sphinxlineitem{Returns:}
\sphinxAtStartPar
boolean: true/false

\end{description}

\end{fulllineitems}

\index{initializeTable() (in module pairing)@\spxentry{initializeTable()}\spxextra{in module pairing}}

\begin{fulllineitems}
\phantomsection\label{\detokenize{pairing:pairing.initializeTable}}
\pysigstartsignatures
\pysiglinewithargsret{\sphinxcode{\sphinxupquote{pairing.}}\sphinxbfcode{\sphinxupquote{initializeTable}}}{\sphinxparam{\DUrole{n}{size}}}{}
\pysigstopsignatures
\sphinxAtStartPar
This function initialize the 2D array to false
\begin{description}
\sphinxlineitem{Args:}
\sphinxAtStartPar
size (int): eg. 30

\sphinxlineitem{Returns:}
\sphinxAtStartPar
boolean{[}{]}{[}{]}: True/False

\end{description}

\end{fulllineitems}

\index{isValid() (in module pairing)@\spxentry{isValid()}\spxextra{in module pairing}}

\begin{fulllineitems}
\phantomsection\label{\detokenize{pairing:pairing.isValid}}
\pysigstartsignatures
\pysiglinewithargsret{\sphinxcode{\sphinxupquote{pairing.}}\sphinxbfcode{\sphinxupquote{isValid}}}{\sphinxparam{\DUrole{n}{todayCombo}}\sphinxparamcomma \sphinxparam{\DUrole{n}{table}}}{}
\pysigstopsignatures
\end{fulllineitems}

\index{pairNow() (in module pairing)@\spxentry{pairNow()}\spxextra{in module pairing}}

\begin{fulllineitems}
\phantomsection\label{\detokenize{pairing:pairing.pairNow}}
\pysigstartsignatures
\pysiglinewithargsret{\sphinxcode{\sphinxupquote{pairing.}}\sphinxbfcode{\sphinxupquote{pairNow}}}{\sphinxparam{\DUrole{n}{numOfStudent}}}{}
\pysigstopsignatures
\end{fulllineitems}

\index{producePair() (in module pairing)@\spxentry{producePair()}\spxextra{in module pairing}}

\begin{fulllineitems}
\phantomsection\label{\detokenize{pairing:pairing.producePair}}
\pysigstartsignatures
\pysiglinewithargsret{\sphinxcode{\sphinxupquote{pairing.}}\sphinxbfcode{\sphinxupquote{producePair}}}{\sphinxparam{\DUrole{n}{classNum}}}{}
\pysigstopsignatures
\sphinxAtStartPar
This function creates random pairs of people that haven’t been previously paired
\begin{description}
\sphinxlineitem{Args:}
\sphinxAtStartPar
classNum (str): eg. “1”

\sphinxlineitem{Returns:}
\sphinxAtStartPar
String: paired names in string

\end{description}
\begin{enumerate}
\sphinxsetlistlabels{\arabic}{enumi}{enumii}{}{.}%
\item {} 
\sphinxAtStartPar
Create new pair

\item {} 
\sphinxAtStartPar
if pairing did not satisfy, try again

\item {} 
\sphinxAtStartPar
record the array of satisfied combos to middle.txt

\item {} 
\sphinxAtStartPar
return the string of paired names

\end{enumerate}

\end{fulllineitems}

\index{record() (in module pairing)@\spxentry{record()}\spxextra{in module pairing}}

\begin{fulllineitems}
\phantomsection\label{\detokenize{pairing:pairing.record}}
\pysigstartsignatures
\pysiglinewithargsret{\sphinxcode{\sphinxupquote{pairing.}}\sphinxbfcode{\sphinxupquote{record}}}{\sphinxparam{\DUrole{n}{validCombo}}\sphinxparamcomma \sphinxparam{\DUrole{n}{table}}}{}
\pysigstopsignatures
\end{fulllineitems}

\index{recordValidCombo() (in module pairing)@\spxentry{recordValidCombo()}\spxextra{in module pairing}}

\begin{fulllineitems}
\phantomsection\label{\detokenize{pairing:pairing.recordValidCombo}}
\pysigstartsignatures
\pysiglinewithargsret{\sphinxcode{\sphinxupquote{pairing.}}\sphinxbfcode{\sphinxupquote{recordValidCombo}}}{\sphinxparam{\DUrole{n}{classNum}}\sphinxparamcomma \sphinxparam{\DUrole{n}{validCombo}}\sphinxparamcomma \sphinxparam{\DUrole{n}{placement}}}{}
\pysigstopsignatures
\sphinxAtStartPar
This function records combo into database\textless{}classNum\textgreater{}.txt or middle\textless{}classNum\textgreater{}.txt
\begin{description}
\sphinxlineitem{Args:}
\sphinxAtStartPar
classNum (str): eg. “1”
validCombo (2D array): eg. {[}{[}4,1{]},{[}7,10{]}….{]}
placement (str): eg. “database”, “middle”

\sphinxlineitem{Returns:}
\sphinxAtStartPar
add new pairings as True values

\end{description}

\end{fulllineitems}


\sphinxstepscope
\index{module@\spxentry{module}!window@\spxentry{window}}\index{window@\spxentry{window}!module@\spxentry{module}}\index{openClass() (in module window)@\spxentry{openClass()}\spxextra{in module window}}\phantomsection\label{\detokenize{window:module-window}}

\begin{fulllineitems}
\phantomsection\label{\detokenize{window:window.openClass}}
\pysigstartsignatures
\pysiglinewithargsret{\sphinxcode{\sphinxupquote{window.}}\sphinxbfcode{\sphinxupquote{openClass}}}{\sphinxparam{\DUrole{n}{classNum}}}{}
\pysigstopsignatures
\end{fulllineitems}

\index{openList() (in module window)@\spxentry{openList()}\spxextra{in module window}}

\begin{fulllineitems}
\phantomsection\label{\detokenize{window:window.openList}}
\pysigstartsignatures
\pysiglinewithargsret{\sphinxcode{\sphinxupquote{window.}}\sphinxbfcode{\sphinxupquote{openList}}}{\sphinxparam{\DUrole{n}{classNum}}}{}
\pysigstopsignatures
\end{fulllineitems}



\chapter{Indices and tables}
\label{\detokenize{index:indices-and-tables}}\begin{itemize}
\item {} 
\sphinxAtStartPar
\DUrole{xref,std,std-ref}{genindex}

\item {} 
\sphinxAtStartPar
\DUrole{xref,std,std-ref}{modindex}

\item {} 
\sphinxAtStartPar
\DUrole{xref,std,std-ref}{search}

\end{itemize}


\renewcommand{\indexname}{Python Module Index}
\begin{sphinxtheindex}
\let\bigletter\sphinxstyleindexlettergroup
\bigletter{c}
\item\relax\sphinxstyleindexentry{countDown}\sphinxstyleindexpageref{countDown:\detokenize{module-countDown}}
\indexspace
\bigletter{e}
\item\relax\sphinxstyleindexentry{ExcelFunctions}\sphinxstyleindexpageref{ExcelFunctions:\detokenize{module-ExcelFunctions}}
\indexspace
\bigletter{g}
\item\relax\sphinxstyleindexentry{GUI}\sphinxstyleindexpageref{GUI:\detokenize{module-GUI}}
\indexspace
\bigletter{p}
\item\relax\sphinxstyleindexentry{pairing}\sphinxstyleindexpageref{pairing:\detokenize{module-pairing}}
\indexspace
\bigletter{w}
\item\relax\sphinxstyleindexentry{window}\sphinxstyleindexpageref{window:\detokenize{module-window}}
\end{sphinxtheindex}

\renewcommand{\indexname}{Index}
\printindex
\end{document}